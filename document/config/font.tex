% ====================================================
% 中文支持,使用 xelatex 编译
% ====================================================
\RequirePackage[fontset=none,scheme=plain]{ctex}

% CJK 衬线字体,影响 \rmfamily 和 \textrm 的字体
\setCJKmainfont{\BOOK@serifCn}[
    BoldFont = \BOOK@serifCnBold,
]
% CJK 无衬线字体,影响 \sffamily 和 \textsf
\setCJKsansfont{\BOOK@sansCn}[
    BoldFont = \BOOK@sansCnBold
]
% CJK 等宽字体,影响 \ttfamily 和 \texttt
\setCJKmonofont{\BOOK@monoCn}[
    BoldFont = \BOOK@monoCnBold
]
% 英文衬线字体,影响 \rmfamily 和 \textrm 的字体
\setmainfont{\BOOK@serif}[
    BoldFont = \BOOK@serifBold,
    ItalicFont = \BOOK@serifItalic,
    BoldItalicFont = \BOOK@serifBoldItalic
]
% 英文无衬线字体,影响 \sffamily 和 \textsf
\setsansfont{\BOOK@sans}[
    BoldFont = \BOOK@sansBold,
    ItalicFont = \BOOK@sansItalic,
    BoldItalicFont = \BOOK@sansBoldItalic
]
% 英文等宽字体,影响 \ttfamily 和 \texttt
\setmonofont{\BOOK@mono}[
    BoldFont = \BOOK@monoBold,
    ItalicFont = \BOOK@monoItalic,
    BoldItalicFont = \BOOK@monoBoldItalic
]

% 声明新的 CJK 字体族
\setCJKfamilyfont{kaishu}{\BOOK@kaishu}
\setCJKfamilyfont{fangsong}{\BOOK@fangsong}

\newcommand*\kaishu{\CJKfamily{kaishu}}
\newcommand*\fangsong{\CJKfamily{fangsong}}